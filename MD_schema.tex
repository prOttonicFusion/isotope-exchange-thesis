\documentclass[a4paper,10pt]{article}
\usepackage[utf8]{inputenc}
\usepackage{tikz,bm}
\usetikzlibrary{matrix,shapes,arrows,positioning,chains}

\begin{document}

\begin{figure}
\begin{center}
% Define block styles
\tikzset{
block/.style={
    rectangle,
    draw,
    text width=35em,
    minimum height=1cm,
    text centered,
    rounded corners
},
connector/.style={
    -latex,
    font=\scriptsize
},
rectangle connector/.style={
    connector,
    to path={(\tikztostart) -- ++(#1,0pt) \tikztonodes |- (\tikztotarget) },
    pos=0.5
},
rectangle connector/.default=-0.6cm,
straight connector/.style={
    connector,
    to path=--(\tikztotarget) \tikztonodes
}}
\begingroup  % compress equations
\medmuskip=2mu
\thinmuskip=1mu
\thickmuskip=2mu
\begin{tikzpicture}
\matrix (m)[matrix of nodes, column  sep=2cm,row  sep=8mm, align=center, nodes={rectangle,draw, anchor=center} ]{
    |[block]| {Ge atomerna utgångspositioner $q^{(0)}$ och hastigheter $\bm{v}^{(0)}$, välj ett tidssteg $dt$ samt återställ $\bm{a}=0$, $t=0$ och $j=0$}              &  \\
    |[block]| {Uppskattningsfas: uppskatta atomernas nya positioner; \\
      Flytta på atomerna: $q^{u} = q^{j} + \bm{v}^{j}dt + \bm{a}\frac{1}{2}dt^2 + ...$\\
      Uppdatera hastigheterna: $\bm{v}^{u}=\bm{v}^{f}+\bm{a}dt + ...$}              &  \\
   |[block]| {Beräkna krafterna på varje atom som $\bm{F} = -\bm{\nabla} (q^{u})$ eller $\bm{F} = \bm{F}(\Psi(q^{u}))$ och $\bm{a}=\bm{F}/m$}              &  \\
   |[block]| {Korrektionsfas: justera atomernas positioner på basis av nya $\bm{a}$;\\
     Flytta på atomerna: $q^{j+1} = q^{u} + $ en funktion av $(\bm{a},dt)$\\
      Uppdatera hastigheterna: $\bm{v}^{j+1}=\bm{v}^{u}+$ en funktion av $(\bm{a},dt)$}              &  \\    
   |[block]| {Tillämpa gränsvilkor, temperatur- samt tryckkontroll vid behov}              &  \\
   |[block]| {Beräkna och skriv ut fysikaliska storheter av intresse}              &  \\
   |[block]| {Skjut frammåt simulationstiden och iterationsantalet: $t=t+dt$, $j=j+1$}              &  \\
   |[block]| {Upprepa så många gånger som det behövs}              &  \\
};
\path [>=latex,->] (m-1-1) edge (m-2-1);
\path [>=latex,->] (m-2-1) edge (m-3-1);
\path [>=latex,->] (m-3-1) edge (m-4-1);
\path [>=latex,->] (m-4-1) edge (m-5-1);
\path [>=latex,->] (m-5-1) edge (m-6-1);
\path [>=latex,->] (m-6-1) edge (m-7-1);
\path [>=latex,->] (m-7-1) edge (m-8-1);
\path [rectangle connector] (m-8-1.west) edge (m-2-1.west);


\end{tikzpicture}
\endgroup
\caption{Ett schema över en typisk MD-simulation. Indexet $u$ visar att storhetens värde är en uppskattning.}
\end{center}
\end{figure}

\end{document}

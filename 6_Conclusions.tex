% ===================================================================================================
\chapter{Conclusions}
% ===================================================================================================

In this thesis, we have studied hydrogen isotope exchange in crystal defects of bulk tungsten using molecular dynamics. 
Due to their high prevalence in irradiated materials, our simulations targeted hydrogen accumulated into mono- and divacancies, dislocations and grain boundaries.  

Our results indicated significantly improved tritium removal rates for all considered defect types when isotope exchange was employed compared to pure diffusion-based removal methods.
Being well in line with earlier experimental work in the field \cite{ahlgren2019hydrogen}, our results provide an atom-level view into the isotope exchange mechanism, showing that it indeed is based on keeping the defect saturated, with loosely bound hydrogen atoms available through the entire annealing process.

Finally, our results serve to prove that it is possible to study the mechanism of isotope exchange using molecular dynamics methods, despite the limitations it poses on simulation time scales and system sizes.
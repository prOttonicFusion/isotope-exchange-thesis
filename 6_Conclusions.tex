% ===================================================================================================
\chapter{Conclusions}
% ===================================================================================================

In this thesis, we have studied hydrogen isotope exchange in crystal defects of bulk tungsten using molecular dynamics. 
Due to their high prevalence in irradiated materials, our simulations targeted hydrogen accumulated into mono- and divacancies, dislocations and grain boundaries.  

Our results indicated significantly improved tritium removal rates for all considered defect types when isotope exchange was employed, compared to pure diffusion-based removal methods at temperatures of 500 K and above.
Through comparison of T-removal rates normalised based on the initial hydrogen population of the defect, we managed to show that the removal occurs fastest from monovacancies (0.01 atoms/ns) and slowest from grain boundaries (0.0017 atoms/ns), with dislocations falling in between (0.003 atoms/ns).

Being well in line with earlier experimental work in the field \cite{ahlgren2019hydrogen}, our results provide an atom-level view into the isotope exchange mechanism, showing that it indeed is based on keeping the defect saturated, with loosely bound hydrogen atoms available through the entire annealing process.

Finally, our results serve to prove that it is possible to study the mechanism of isotope exchange using molecular dynamics methods, despite the limitations it poses on simulation time scales and system sizes.
With simulation times ranging up to microseconds and a temporal resolution of the order of picoseconds, MD provides an excellent tool for bridging the gap between micro- and macroscopic studies.

In the future, the results of this study could be extended through a study of impurity atoms in tungsten, as well as their effects on hydrogen trapping and removal.
This would, however, require a reliable and efficient potential to be developed e.g. for a W-H-C system -- a task far beyond the scope of this thesis.

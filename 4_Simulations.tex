% ===================================================================================================
\chapter{Simulations}
% ===================================================================================================

The molecular dynamics code used in this work was the Large-scale Atomic/Molecular Massively Parallel Simulator (LAMMPS) developed by Sandia National Laboratories \cite{lammpsMD}. 
LAMMPS is a free and open-source code, utilizing the Message Passing Interface (MPI) for running spatially parallelized parallelized MD simulations.
In addition, LAMMPS also provides a scripting language, which can be used to set input parameters and run simulations, but also contains tools for creating and manipulating lattices as well as simple simulation cells.    

The interatomic potential of choice for most of our simulations was the W-H-He EAM potential developed by Bonny \textit{et al.} and referred to as 'EAM1' in \cite{bonny2014binding}. 
The W-W interaction part of this potential is the same as 'EAM2' developed by Marinica \textit{et al.} in \cite{marinica2013interatomic} and well-known for providing elastic constants and point defect, dislocation and grain boundary properties in good agreement with Density Functional Theory (DFT) calculations and experiments \cite{bonny2014many}. 
The W-H-He potential, referred hereafter to as just 'EAM1', has been shown to relatively accurately recreate the interactions of H with both point defects and dislocations in W \cite{bonny2014binding, grigorev2015interaction}. 


% ---------------------------------------------------------------------------------------------------
\section{Simulation Setups}
% ---------------------------------------------------------------------------------------------------
Although the simulations differ largely from defect to defect, a general pattern can be extracted for all isotope exchange simulations. 
First, a simulation cell of bulk W containing the defect was created and allowed to reach a minimum energy configuration by first using numerical minimization and then relaxation through repeatedly heating and cooling the system during short MD cycles. 
The method of energy minimzation used in all simulations is a Polak-Ribi\`{e}re conjugate gradient algorithm \cite{polak1969note}, used to iteratively adjust atom coordinates until a local energy minimum was found. 

Following the initial relaxation of the simulation cell, a large amount of T atoms were deposited at random positions around the cell and allowed to diffuse around during successive MD runs, until the defect had been saturated with hydrogen.
After this, the excess T was removed prior to adding H to random TIS positions around the bulk material. 
Isotope exchange is then simulated by running MD for a total simulation time of 100...1450 ns. 
Periodic boundaries are applied in all directions and the pressure and temperature are controlled using a Nos\'{e}-Hoover thermostat and barostat, emulating an isothermal-isobaric ensemble. 
In a real world situation, a T atom leaving a defect and diffusing far away will very unlikely return to the same defect, instead of getting caught in another defect or leaving the material through a surface. 
Due to the periodic boundaries, however, this behavior is not seen as the T atom exiting the cell simply returns from the opposite side. 
T counter this, the simulation is performed in intervals of 5 ps (5000 time steps) and between each interval, any T atoms having moved 'far away' are removed from the simulation. 
This decision is made based on the current distance of each T atom from its initial position. 
If the distance exceeds a threshold of $d_{\rm{rmv}} =$ 14 \AA, i.e. ca 4.5 unit cells, the T atom is considered to have left the system. 




% --------------- IsoEx MD ------------------------------------------------------------------------------------
\begin{figure}[!ht]
\begin{center}
% Define block styles
\tikzset{
decision/.style = {diamond, draw, fill=green!40, 
    text width=4.5em, text badly centered, node distance=3cm, inner sep=0pt},
block/.style = {rectangle, draw, fill=cyan!30, 
    text width=12em, text centered, rounded corners, minimum height=2em},
smallblock/.style = {block, text width=7em},
cloud/.style = {draw, ellipse,fill=red!20, node distance=3cm,
    minimum height=2em}
    }
\begingroup  % compress equations
\medmuskip=2mu
\thinmuskip=1mu
\thickmuskip=2mu
\begin{tikzpicture}
\matrix (m)[matrix of nodes, column  sep=0.0cm,row  sep=5mm, align=center, nodes={rectangle,draw, anchor=south} ]{
   |[block] (initI)| {Create simulation cell containing defect} & \\
   |[block] (initII)| {Relax \& Deposit H} & \\
    |[block] (MD)| {Run MD for 5000 steps}          &  \\
    |[block] (DelT)| {Delete T atoms for which $d_i > d_{\text{rmv}}$}          &  \\
%    |[block] (Calcd)| {Calculate $d_i$}          &  \\
%    |[decision] (IsOut)| {$d_i > d_{\rm{rmv}}$?}              &  \\
%       & |[smallblock] (Rmv)| {Remove $i$th T}         &  \\
   |[decision] (IsEnd)| {End condition reached?}              &  \\
   |[block] (End)| {Save results \& quit}   & \\
};
\path [>=latex,->] (initI) edge (initII);
\path [>=latex,->] (initII) edge (MD);
\path [>=latex,->] (MD) edge (DelT);
\path [>=latex,->] (DelT) edge (IsEnd);
%\path [>=latex,->] (MD) edge (Calcd);
%\path [>=latex,->] (Calcd) edge (IsOut);
%\draw [>=latex,->] (IsOut.east) -| node[above, near start] {yes} (Rmv.north);
%\draw [>=latex,->] (IsOut.south) -- node[right] {no}(IsEnd);
%\draw [>=latex,->] (Rmv.south) |- (IsEnd.north);
\draw [>=latex,->] (IsEnd.west) -- node[above] {no} ++(-2,0cm) |- (MD.west);
\draw [>=latex,->] (IsEnd.south) -- node[right] {yes}(End);
\end{tikzpicture}
\endgroup
\caption{A flowchart representation of the isotope exchange simulations. Variable $d_i$ refers to the distance between current and initial point of T atom $i$.} 
\label{Fig:isoExSimus}
\end{center}
\end{figure}


% ---------------------------------------------------------------------------------------------------
\subsection{Vacancies}
% ---------------------------------------------------------------------------------------------------
In the vacancy case, a monovacancy was created by removing the middlemost atom of a $10\times 10 \times 10$ unit cell (2000 atom) W lattice.  
In contrast to the larger defects, where the defect was saturated using simulated diffusion, the small size of the point defect enabled us to manually place T at their natural lowest energy positions, i.e. at the octahedral interstitial sites, forming a square bipyramid around the vacancy \cite{heinolaTungstenDFT}.
A total of 19 H atoms were then deposited to randomly chosen tetragonal interstitial sites around the simulation W lattice, as seen in fig. (\ref{Fig:monovac_system}), bringing the (H+T)/W ratio to 0.0125. 
The T/H ratio, on the other hand, is 0.32.

A divacancy system was constructed in a similar fashion by removing two neighboring W atoms and adding a total of 10 T atoms to the defect.

\begin{figure}[!ht]
\center
\includegraphics[width=0.7\linewidth]{1Vac_system.png}
\caption{The initial state of the simulation cell used in the monovacancy simulations. 
The W atoms are rendered translucent to display the positions of the hydrogen isotopes.}
\label{Fig:monovac_system}
\end{figure}

% ---------------------------------------------------------------------------------------------------
\subsection{Dislocations}
% ---------------------------------------------------------------------------------------------------
In the dislocation case, we have created a 1/2\hkl[1 1 1]\hkl{1 0 0} edge dislocation, in a $109.0 \times 136.8 \times 18.4$ $\AA^3$ tungsten super cell with the $x$, $y$ and $z$ axes oriented along the crystal directions  \hkl[1 1 1], \hkl[1 1 -2] and \hkl[-1 1 0] respectively, as shown in fig. (\ref{Fig:disloc_system}). 
The supercell was then divided into three equally thick slices, parallel to the $xz$ plane and a dislocation introduced through the addition of a \hkl{1 1 1} atom plane (parallel to the $yz$ plane) from to the middlemost slice. 
In order to facilitate the formation of a natural dislocation and to enable the use of periodical boundary conditions, the atom positions in the middle slice were finally compressed slightly.

% [Skiss of the dislocation manufacturing process?]

\begin{figure}[!ht]
\center
\includegraphics[width=0.7\linewidth]{disloc_system.png}
\caption{The initial state of the simulation cell used in the dislocation simulations. 
The W atoms are rendered translucent to display the positions of the hydrogen isotopes. 
The dislocation, together with its periodic images, is visualized using a defect mesh.}
\label{Fig:disloc_system}
\end{figure}

% ---------------------------------------------------------------------------------------------------
\subsection{Grain Boundaries}
% ---------------------------------------------------------------------------------------------------
% $\Sigma$5\hkl{310}/\hkl[001]
For simulating isotope exchange in W grain boundaries, we used a system containing an arbitrarily chosen \hkl(310)\hkl[001] tilt grain boundary. The simulation cell was created by growing together two separate tungsten lattices, each rotated through $\pm18.43^\circ$, respectively, around the $z$-axis. 
The $x$-axes of the lattices now point along directions of \hkl<310> and we have a system with a periodicity of $\sqrt{10}~a$ along the $x$- and $y$-axes, and $a$ along the $z$-axis. 
In order to minimize the number of atoms needed to simulate the system we can set $y >> x \approx z$ and again use periodic boundaries.

% Zhou_2009_H_behaviour_in_W_grain_boudnary_FP.pdf

\begin{figure}[!ht]
\center
\includegraphics[width=0.94\linewidth]{GB_system.png}
\caption{The initial state of the simulation cell used in the grain boundary simulations. 
The W atoms are rendered translucent to display the positions of the hydrogen isotopes.}
\label{Fig:GB_system}
\end{figure}

% ---------------------------------------------------------------------------------------------------
%\section{Impurities}
% ---------------------------------------------------------------------------------------------------


% ===================================================================================================
\chapter{Molecular Dynamics Simulations}
% ===================================================================================================

% ---------------------------------------------------------------------------------------------------
\section{Molecular Dynamics in General}
% ---------------------------------------------------------------------------------------------------
Molecular dynamics (MD) as a simulation method originates from the 1950s and is therefore one of the oldest tools used in computational physics and materials research \cite{alder1957phase}.
On a fundamental level, MD consists of solving complex many-body problems iteratively to determine the forces acting on individual particles. 
These calculations, too complex  to utilize any analytical solution, are performed using numerical methods and allow us to relatively accurately simulate the dynamic evolution of (semi-)classical systems of particles. 

The basic algorithm of a MD simulation is described in fig. \ref{MD-schema}. 
The simulation is initiated by giving each atom, $i=1,...,N$, an initial position $\mathbf{r}_i^{(0)}$ and random velocity $\mathbf{v}_i^{(0)}$, consistent with the desired net temperature of the system. 
The positions and velocities of each particle are then updated by solving Newton's equations of motion
\begin{align}
\mathbf{F}_i(\mathbf{r}_i,t) = m_i\frac{d^2\mathbf{r}_i}{dt^2} = m_i\mathbf{a}_i(t) = -\nabla_{r_i}V(\mathbf{r}_i)
\end{align}
where $m_i$ is the mass of atom i, positioned at $\mathbf{r}_i$. 
The force $\mathbf{F}_i$ acting on each particle is determined by the interatomic potential $V(\mathbf{r}_i)$ and can be used to calculate the acceleration of each particle. 
The new positions and velocities are then simply functions of the acceleration and the simulation time step $\Delta t$. 
The simulation then proceeds through repeated evaluations of the equations of motion, updating the system in increments of $\Delta t$ time units. 
In order to avoid catastrophic errors in the conservation of energy, the time step must be shorter than the inverse vibrational frequency of any process in the system, typically around 1 fs.

% [Neighbor lists & force cutoff]
In the simplest form of a molecular dynamics algorithm, the number of calculations performed scales as $O(N^2)$, where $N$ is the number of atoms. 
In a typical simulation, however, the particles only interact with a limited number of neighbors and long range interactions can be safely ignored. 
The most typical way of doing this is by declaring a cutoff radius for the force calculations, with only particles within this radius from particle $i$ taken into account when evaluating $\mathbf{F}_i(\mathbf{r}_i,t)$. 
This theoretically reduces the scaling of the algorithm to $O(N)$.

% [Periodic boundaries] 
Despite the use of a force cutoff, most real materials are still far too large to be simulated accurately.
For example, a typical metal has a number density of around $5\cdot 10^{22}$ atoms/cm$^3$, meaning that simulating even just a $100\times 100 \times 100$ nm$^3$ block of metal, generally considered to be the lower limit of a bulk material, would require us to keep track of $5\cdot 10^7$ atoms.
A common way of 'cheating' is through the use of \textit{periodic boundary conditions}. 
This means that particles exiting the region of space confined by the simulation cell borders re-appear on the opposite side, as visualized in fig. (\ref{Fig:pbcs}). 
Since also interatomic forces are allowed to interact over cell borders, we effectively have the cell surrounded by infinitely many copies of itself.

\begin{figure}[!ht]
\center
\includegraphics[width=0.75\linewidth]{pbc.png}
\caption{A visualization of periodic boundary conditions in a 2D simulation cell. The cells with dashed lines represent virtual copies}
\label{Fig:pbcs}
\end{figure}

With a perfect lattice, a simulation cell with periodic boundary conditions accurately emulates a bulk material. 
With various defects introduced, we, however, have to make sure that the cell is large enough for the created strain field not to overlap with itself, possibly affecting the results.

% [Thermostats, barostats]
Real systems are seldom isolated and exchange heat with their surroundings in a number of ways. 
In order to sample a non-isolated thermodynamic ensemble, we need to manually control the temperature of our simulation. 
Temperature of an atomic system is defines as
\begin{align}
T(t) = \frac{1}{k_{\text{B}}N_f}\sum_{i}\left[ m_iv_{i,x}^2(t) + m_iv_{i,y}^2(t) + m_iv_{i,z}^2(t)\right]
\end{align}
where $N_f$ is the number of degrees of freedom and $v_{i}$ is the velocity of atom $i$. 
It is therefore easy to see how we can control the temperature by scaling atomic velocities.
A common and relatively simple way of doing this is called the \textit{Berendsen thermostat} and defined as
\begin{align}
\frac{dT}{dt} = \frac{T_0-T}{\tau}
\end{align}
where $T_0$ is the target temperature and $\tau$ a time constant \cite{berendsen1984molecular}. 
Despite its high efficiency, the Berendsen thermostat is not actually producing a correct canonical ensemble and is prone to causing various artifacts.

The \textit{Nos\'{e}-Hoover thermostat}, on the other hand, accurately samples the canonical ensamble \cite{nose1984unified}. 
Being a so called \textit{extended system method}, the Nos\'{e}-Hoover thermostat makes use of a virtual particle reservoir, acting as a heat bath. 
Since this requires equations of motions to be evaluated both for the actual particles and the virtual reservoir, the gained physical accuracy comes at higher computational cost.

% --------------- MD schematic ------------------------------------------------------------------------------------
\begin{figure}
\begin{center}
% Define block styles
\tikzset{
decision/.style = {diamond, draw, fill=green!40, 
    text width=4.5em, text badly centered, node distance=3cm, inner sep=0pt},
block/.style = {rectangle, draw, fill=cyan!30, 
    text width=20em, text centered, rounded corners, minimum height=2em},
cloud/.style = {draw, ellipse,fill=red!20, node distance=3cm,
    minimum height=2em}
    }

\begingroup  % compress equations
\medmuskip=2mu
\thinmuskip=1mu
\thickmuskip=2mu
\begin{tikzpicture}
\matrix (m)[matrix of nodes, column  sep=1.5cm,row  sep=5mm, align=center, nodes={rectangle,draw, anchor=center} ]{
    |[block]| {Set initial conditions $\mathbf{r}_i(t_0)$ and $\mathbf{v}_i(t_0)$, set a timestep $dt$ and reset $\mathbf{a}=0$, $_0t=0$}              &  \\
    %|[block]| {\textit{Estimation step:}\\ Estimate new atom positions; \\
    %  Move atoms: $\mathbf{r}_i^{(e)} = \mathbf{r}_i^{(j)} + \mathbf{v}^{(j)}dt + \mathbf{a}\frac{1}{2}dt^2 + ...$\\
    %  Update velocities: $\mathbf{v}^{(e)}=\mathbf{v}^{(f)}+\mathbf{a}dt + ...$}              &  \\
   |[block]| {Calculate $\mathbf{F}_i = -\mathbf{\nabla} V(\mathbf{r}_i)$ and $\mathbf{a}_i=\mathbf{F}_i/m_i$}              &  \\
   |[block]| {Adjust atom positions based on the new $\mathbf{a}_i$;\\
     Move atoms: $\mathbf{r}_i(t_{n+1}) = \mathbf{r}_i(t_n) + f(\mathbf{a}_i,\Delta t)$\\
      Update velocities: $\mathbf{v}(t_{n+1})=\mathbf{v}(t_n)+g(\mathbf{a}_i,\Delta t)$}              &  \\
   |[block]| {Apply boundary conditions, thermostats and barostats as needed}              &  \\
   |[block]| {Calculate and output physical quantities of interest}              &  \\
   |[block]| {$t_{n+1} = t_n+\Delta t$}              &  \\
   |[decision]| {End condition reached?}              &  \\
   |[block]| {Collect data and quit}              &  \\
};
\path [>=latex,->] (m-1-1) edge (m-2-1);
\path [>=latex,->] (m-2-1) edge (m-3-1);
\path [>=latex,->] (m-3-1) edge (m-4-1);
\path [>=latex,->] (m-4-1) edge (m-5-1);
\path [>=latex,->] (m-5-1) edge (m-6-1);
\path [>=latex,->] (m-6-1) edge (m-7-1);
%\path [>=latex,->] (m-7-1) edge (m-8-1);
\draw [>=latex,->] (m-7-1.west) -- node[above] {no} ++(-4,0cm) |- (m-2-1.west);
\draw [>=latex,->] (m-7-1.south) -- node[right] {yes}(m-8-1);
\end{tikzpicture}
\endgroup
\caption{A flowchart representation of a typical Molecular Dynamics simulation. Above, $f$ and $g$ refer to two different functions of the timestep and the acceleration. In the most simple case these would be $f(\mathbf{a}_i,\Delta t) = \mathbf{v}_i\Delta t + \mathbf{a}_i\Delta t^2$ and $g(\mathbf{a}_i,\Delta t) = \mathbf{a}_i\Delta t$, respectively.} 
\label{MD-schema}
\end{center}
\end{figure}
%------------------------------------------------------------------------------------------------------------------


\subsection{Interatomic potentials}

In an MD simulation, potential energies of simulated particles are calculated using so called interatomic potentials - mathematical functions designed to approximately emulate the interactions between atoms. 
These potentials are usually based upon the Born-Oppenheimer model, i.e. that the electrons of alla atoms are permanently in the ground state and all interactions depend purely on the interatomic distance \cite{born1927quantentheorie}. 
In general terms, the potential energy of an atomistic system can be written as
\begin{align}
V_{tot} = \sum_i^N V_1(\mathbf{r}_i) + \sum_{i,j}^N V_2(\mathbf{r}_i, \mathbf{r}_j) +  \sum_{i,j,k}^N V_3(\mathbf{r}_i, \mathbf{r}_j, \mathbf{r}_k) + ...
\label{V_tot-ekv}
\end{align}
where $N$ is the number of atoms in the system and $V_1$, $V_2$ and $V_3$ respectively are one, two and three body terms \cite{potentialsTheory}.

The indices $i$, $j$ and $k$ iterate through the atom positions in three spatial dimensions and can be restricted to $i < j$ and $j < k$ for symmetric interactions such as monoelemental systems. 
The first term of eq. (\ref{V_tot-ekv}) can be discarded for systems not affected by an external field, an we are thus left with
\begin{align}
V_{tot} = \sum_i \sum_{j>i} V_2(\mathbf{r}_i, \mathbf{r}_j) + \sum_i \sum_{j>i} \sum_{k > j} V_3(\mathbf{r}_i, \mathbf{r}_j, \mathbf{r}_k) + ...
\end{align}
Potentials employing terms of a higher order than two are referred to as many-body potentials, while those using only the two first terms are pair potentials.

The atomic structure of metals allows the to be described particularly well using the so called \textit{Embedded Atom Method} (EAM) formalism, giving the potential energy in the form
\begin{align}
V_{tot} = \sum_i^N F_i\, \bigg( \sum_j \rho\, (\mathbf{r}_{ij}) \bigg) + \frac{1}{2} \sum^N_{ij} V_2 (\mathbf{r}_{ij})
\end{align}
where  $F_i$ is a function of the summed electron density $\rho (\mathbf{r}_{ij})$ and $\mathbf{r}_{ij}$ denotes the distance $| \mathbf{r}_i - \mathbf{r}_j |$ between the $i$th and $j$th atoms. \cite{EAMmodel,dudarevEAMpotential}. 

For covalently bonded materials, is is often more accurate to use a \textit{Bond Order Potential} (BOP), generally presented as
\begin{align}
V_{ij}(r_{ij}) =bV_{ijk}V_{\rm attractive}(r_{ij}) V_ {\rm repulsive}
\end{align}
Examples of BOPs are potentials of Tersoff, Brenner and Finnis-Sinclair type. \cite{tersoff1988new, brenner1990empirical, finnis1984simple} Both BOP and EAM potentials have the functional shape of a typical pair-potential, but act as many-body potentials due to many-body interactions embedded into the pair-terms.

\section{Parallel Algorithms}
As we all know, the volume of a cube is related to the its side length as $V=a^3$. 
Applying this to atomic systems, this means that if we double the side length of a cubical simulation cell, the number of atoms increases by a factor of eight. 
Correspondingly, the computational load also increases by a factor of between 8 and 64, depending on the use and size of neighbor lists.

% ---------------------------------------------------------------------------------------------------
\section{Analysis of the Results}
% --------------------------------------------------------------------------------------------------- 
For each successful simulation, two coordinate dump files were produced, one containing the coordinates of the H and T atoms every 250 time steps ($2.5\cdot 10^{-4}$ ns) and another with the W atom coordinates every $5\cdot 10^5$ time steps (0.5 ns).
Since the motion of the defects themselves was practically negligible, the number of H and T bound to a defect could be determined by defining a geometric region in space around the defect and overlaying this region with each frame of the H-T dump file.

In practice, this was performed by parsing the dump files with a simple Fortran program, which checked whether each atom was located inside or outside the specified region and outputted the number of H and T in the defect as a function of time.

% ===================================================================================================
\chapter{Simulations}
% ===================================================================================================

We have studied isotope exchange in four common crystallographic defect systems in tungsten. 
Regardless of the simulated defect, each simulation followed the same general pattern. 
First, a simulation cell of bulk W containing the defect was created and allowed to reach a minimum energy configuration by first using numerical minimization and then relaxation through repeatedly heating and cooling the system during short MD cycles. 
The method of energy minimzation used in all simulations is a Polak-Ribi\`{e}re conjugate gradient algorithm \cite{polak1969note}, used to iteratively adjust atom coordinates until a local energy minimum was found. 

After the initial relaxation, the defect was saturated with T during a 10...200 ns MD run and the excess was removed, prior to adding H to random TIS positions around the bulk material. 
Isotope exchange is then simulated by running MD for a total simulation time of 100...1450 ns. 
Periodic boundaries are applied in all directions and the pressure and temperature are controlled using a Nos\'{e}-Hoover thermostat and barostat, emulating an isothermal-isobaric ensemble. 
In a real world situation, a T atom leaving a defect and diffusing far away will very unlikely return to the same defect, instead of getting caught in another defect or leaving the material through a surface. 
Due to the periodic boundaries, however, this behavior is not seen as the T atom exiting the cell simply returns from the opposite side. 
T counter this, the simulation is performed in intervals of 5 ps (5000 time steps) and between each interval, any T atoms having moved 'far away' are removed from the simulation. 
This decision is made based on the current distance of each T atom from its initial position. 
If the distance exceeds a threshold of $d_{\rm{rmv}} =$ 14 \AA, i.e. ca 4.5 unit cells, the T atom is considered to have left the system. 




% --------------- IsoEx MD ------------------------------------------------------------------------------------
\begin{figure}
\begin{center}
% Define block styles
\tikzset{
decision/.style = {diamond, draw, fill=green!40, 
    text width=4.5em, text badly centered, node distance=3cm, inner sep=0pt},
block/.style = {rectangle, draw, fill=cyan!30, 
    text width=12em, text centered, rounded corners, minimum height=2em},
smallblock/.style = {block, text width=5em},
cloud/.style = {draw, ellipse,fill=red!20, node distance=3cm,
    minimum height=2em}
    }
\begingroup  % compress equations
\medmuskip=2mu
\thinmuskip=1mu
\thickmuskip=2mu
\begin{tikzpicture}
\matrix (m)[matrix of nodes, column  sep=0.0cm,row  sep=5mm, align=center, nodes={rectangle,draw, anchor=south} ]{
   |[block] (initI)| {Create simulation cell containing defect} & \\
   |[block] (initII)| {Relax \& add hydrogen} & \\
    |[block] (MD)| {Run MD for 5000 time steps}          &  \\
    |[block] (Calcd)| {Calculate $d_i$}          &  \\
   |[decision] (IsOut)| {$d_i > d_{\rm{rmv}}$?}              &  \\
       & |[smallblock] (Rmv)| {Remove $i$th T atom}         &  \\
   |[decision] (IsEnd)| {End condition reached?}              &  \\
   |[block] (End)| {Save results \& quit}   & \\
};
\path [>=latex,->] (initI) edge (initII);
\path [>=latex,->] (initII) edge (MD);
\path [>=latex,->] (MD) edge (Calcd);
\path [>=latex,->] (Calcd) edge (IsOut);
\draw [>=latex,->] (IsOut.east) -| node[above, near start] {yes} (Rmv.north);
\draw [>=latex,->] (IsOut.south) -- node[right] {no}(IsEnd);
\draw [>=latex,->] (Rmv.south) |- (IsEnd.north);
\draw [>=latex,->] (IsEnd.west) -- node[above] {no} ++(-2,0cm) |- (MD.west);
\draw [>=latex,->] (IsEnd.south) -- node[right] {yes}(End);
\end{tikzpicture}
\endgroup
\caption{A flowchart representation of the isotope exchange simulations. Variable $d_i$ refers to the distance between current and initial point of T atom $i$.} 
\label{Fig:isoExSimus}
\end{center}
\end{figure}


% ---------------------------------------------------------------------------------------------------
\section{Vacancies}
% ---------------------------------------------------------------------------------------------------
In the vacancy case, a monovacancy was created by removing the middlemost W atom of a $10\times 10 \times 10$ unit cell (2000 atom) W lattice. 
As seen in fig. (\ref{Fig:monovac_system}), six T atoms were then added to the vacancy at their lowest energy positions \cite{heinolaTungstenDFT}, i.e. at the octahedral interstitial sites, forming a square bipyramid. 
A total of 19 H atoms were then deposited to randomly chosen tetragonal interstitial sites around the simulation W lattice, bringing the (H+T)/W ratio to 0.0125. 
The T/H ratio, on the other hand, is 0.32.

A divacancy system was constructed in a similar fashion by removing two neighboring W atoms and adding a total of 10 T atoms to the defect.

\begin{figure}[!ht]
\center
\includegraphics[width=0.7\linewidth]{1Vac_system.png}
\caption{The initial state of the simulation cell used in the monovacancy simulations. 
The W atoms are rendered translucent to display the positions of the hydrogen isotopes.}
\label{Fig:monovac_system}
\end{figure}

% ---------------------------------------------------------------------------------------------------
\section{Dislocations}
% ---------------------------------------------------------------------------------------------------
In the dislocation case, we have created a 1/2\hkl[1 1 1]\hkl{1 0 0} edge dislocation, in a $109.0 \times 136.8 \times 18.4$ $\AA^3$ tungsten super cell with the $x$, $y$ and $z$ axes oriented along the crystal directions  \hkl[1 1 1], \hkl[1 1 -2] and \hkl[-1 1 0] respectively, as shown in fig. (\ref{Fig:disloc_system}). 
The supercell was then divided into three equally thick slices, parallel to the $xz$ plane and a dislocation introduced through the addition of a \hkl{1 1 1} atom plane (parallel to the $yz$ plane) from to the middlemost slice. 
In order to facilitate the formation of a natural dislocation and to enable the use of periodical boundary conditions, the atom positions in the middle slice were finally compressed slightly.

% [Skiss of the dislocation manufacturing process?]

\begin{figure}[!ht]
\center
\includegraphics[width=0.7\linewidth]{disloc_system.png}
\caption{The initial state of the simulation cell used in the dislocation simulations. 
The W atoms are rendered translucent to display the positions of the hydrogen isotopes. 
The dislocation, together with its periodic images, is visualized using a defect mesh.}
\label{Fig:disloc_system}
\end{figure}

% ---------------------------------------------------------------------------------------------------
\section{Grain Boundaries}
% ---------------------------------------------------------------------------------------------------
% $\Sigma$5\hkl{310}/\hkl[001]
For simulating isotope exchange in W grain boundaries, we used a system containing an arbitrarily chosen \hkl(310)\hkl[001] tilt grain boundary. The simulation cell was created by growing together two separate tungsten lattices, each rotated through $\pm18.43^\circ$, respectively, around the $z$-axis. 
The $x$-axes of the lattices now point along directions of \hkl<310> and we have a system with a periodicity of $\sqrt{10}~a$ along the $x$- and $y$-axes, and $a$ along the $z$-axis. 
In order to minimize the number of atoms needed to simulate the system we can set $y >> x \approx z$ and again use periodic boundaries.

% Zhou_2009_H_behaviour_in_W_grain_boudnary_FP.pdf

\begin{figure}[!ht]
\center
\includegraphics[width=0.94\linewidth]{GB_system.png}
\caption{The initial state of the simulation cell used in the grain boundary simulations. 
The W atoms are rendered translucent to display the positions of the hydrogen isotopes.}
\label{Fig:GB_system}
\end{figure}

% ---------------------------------------------------------------------------------------------------
%\section{Impurities}
% ---------------------------------------------------------------------------------------------------

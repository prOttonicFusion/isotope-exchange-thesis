% ===================================================================================================
\chapter{Introduction}
% ===================================================================================================

One viable solution to the increasing need for large scale production of clean energy, is the use of thermonuclear fusion. In its most researched form, nuclear fusion consists of the fusing of two hydrogen isotopes, deuterium (D, $^2_1$H) and tritium (T, $^3_1$H) into helium ($^4_2$He) and one neutron, in a process which releases large amounts of energy. Per unit of mass, the D-T reaction offers almost six orders of magnitude more energy than any chemical process, such as burning fossil fuel, and three times more than the fission of uranium-235.  

Most of the energy emitted from the D-T reaction is released as the kinetic energy of the neutron. The plasma facing surfaces of a fusion reactor are therefore subject to a constant bombardment by neutrons moving at over 17 \% of the speed of light, potentially displacing atoms from their lattice positions on impact and causing widespread crystallographic damage. Due to their electrical neutrality, neutrons interact only to a negligible degree with electrons and can thus penetrate several meters into the reactor walls, causing damage deep inside the bulk material. 

Chosen for its high resistance against surface erosion by energetic ions and neutrons, tungsten (W) is the main armor material candidate for the divertor of the ITER tokamak. % [brief description of divertors?]
It has, however, been shown that hydrogen can diffuse easily deep into W, while crystallographic defects, such as those produced by the neutron radiation, increase the retention of hydrogen isotopes within the material. \cite{tanabe2014review} 
Due to the presence of the radioactive isotope tritium in the reactor, plasma-facing components will grow radioactive over time. The inventory of embedded tritium is therefore a critical safety concern, but also a matter of fuel economy, since tritium is extremely rare and among the most expensive substances found on Earth.

Proposed methods for T removal include thermal treating of the components until most H has been detrapped and removed \cite{heinola2017long}, but also the use of high-power pulsed flashlamps \cite{gibson2005removal} or pulsed lasers \cite{skinner2008recent,de2017efficiency} to only locally heat the material. For example, the ITER plan for T removal consists of vacuum annealing the first wall at 513 K and the divertor at 623 K \cite{pitts2011physics}, essentially heating the components long enough for the embedded hydrogen isotopes to diffuse out through the nearest surface.
These methods, however, involves heating massive metal objects to high temperatures for up to several days. In the case of the ITER T removal plan, tests performed at the ITER-like JET reactor suggest that annealing at the proposed temperatures for 15 h, will likely result in a removal of only around 40 \% of the tritium \cite{heinola2017long}. Conveniently, it has been shown that annealing under a H$_2$ atmosphere can achieve the same result of removing tritium, but at lower temperatures. This method makes use of the spontaneous replacement of tritium with a cheaper and safer isotope, protium (H, $^1_1$H), in a process called 'isotope exchange'. Experiments have shown the amount of retained hydrogen isotope, deuterium dropping to nearly zero after 24 h of annealing at 520 K under a H$_2$ atmosphere \cite{ahlgren2019hydrogen}, while a similar removal rate using traditional vacuum annealing would require temperatures of at least 700 K.

Experimental studies of hydrogen isotope exchange in tugsten have been conducted by e.g. Alimov \textit{et al.} \cite{alimov2011hydrogen}, Roth \textit{et al.}  \cite{roth2013hydrogen} and Barton \textit{et al.} \cite{barton2014deuterium}. The phenomenon has also been modelled on a macroscopic scale by Hodille \textit{et al.} \cite{hodille2016study} using Macroscopic Rate Equation models, based on experimental data as well DFT data on the behavior of hydrogen near tungsten monovacancies. In this thesis, we set out to study the isotope exchange on a microscopic scale using molecular dynamics to simulate the interactions with hydrogen and various crystallographic defects in tungsten.

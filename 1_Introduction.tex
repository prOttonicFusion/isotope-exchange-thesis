% ===================================================================================================
\chapter{Introduction}
% ===================================================================================================

Many fundamental challenges of our century can be linked to finding ways of meeting the globally increasing demand for sustainable energy. 
One viable solution to increasing the large scale production of clean energy is through the harnessing of thermonuclear fusion. 
In its most researched form, nuclear fusion consists of fusing two hydrogen isotopes, deuterium (D, $^2_1$H) and tritium (T, $^3_1$H), and thereby producing helium ($^4_2$He) and a neutron, in a process which releases large quantities of energy. 
Per unit of mass, the D-T reaction offers almost six orders of magnitude more energy than any chemical process, including the burning of fossil fuels, and three times more than fission of uranium-235 in a modern nuclear power plant.  

Most of the energy emitted from the D-T reaction is released as the kinetic energy of the neutron. 
The plasma-facing surfaces of a fusion reactor are therefore subject to a constant bombardment by neutrons moving at over 17 \% of the speed of light, having the potential to displace atoms from their lattice positions on impact and cause widespread crystallographic damage. 
Due to their electrical neutrality, neutrons interact only to a negligible degree with electrons, limiting their interactions with atoms to the highly improbable collisions with atomic nuclei. 
Neutrons can thus penetrate several meters into the reactor walls, extending the damage from the surface, deep into the bulk material. 

Chosen for its high resistance against erosion by energetic ions and neutrons, tungsten (W) is the primary armour material candidate for the divertor of the groundbreaking ITER tokamak being constructed in southern France \cite{hirai2013iter}.
It has, however, been shown that hydrogen isotopes can easily diffuse deep into bulk W, while crystallographic defects, such as those produced by the neutron radiation, enhance the retention of hydrogen within the material. \cite{tanabe2014review} 
This trapping mechanism tends to lead to an accumulation of hydrogen, including the radioactive isotope tritium (T), from the fusion fuel into the plasma-facing reactor components, rendering them radioactive over time. 
The inventory of embedded tritium is, therefore, a critical safety concern, but also a matter of fuel economy, since tritium is extremely rare and among the most expensive substances found on Earth.

Proposed methods for T removal include thermal treating of the affected components until most hydrogen has been detrapped and removed through diffusion \cite{heinola2017long}, but also the use of high-power pulsed flashlamps \cite{gibson2005removal} or lasers \cite{skinner2008recent,de2017efficiency} to only locally heat the material. 
For example, in the case of ITER, the plan for T removal consists of vacuum annealing the first wall at 513 K and the divertor at 623 K \cite{pitts2011physics}, essentially heating the components long enough for the embedded hydrogen isotopes to diffuse out through the nearest surface.
In practice, however, this method involves heating massive metal objects -- the divertor alone being a multi-part component built up from 54 ten-tonne cassette assemblies -- to high temperatures for extended periods of time. 
Tests performed at the ITER-like JET reactor additionally suggest that annealing at the proposed temperatures for as long as 15 h, will likely result in the removal of only around 40 \% of the tritium inventory \cite{heinola2017long}. 

Conveniently, it has been shown \cite{alimov2011hydrogen, roth2013hydrogen, barton2014deuterium} that annealing under a H$_2$ atmosphere can achieve similar results of tritium removal, but at significantly lower temperatures. 
This method makes use of the spontaneous replacement of tritium with a cheaper and safer isotope, protium (H, $^1_1$H), in a process aptly referred to as 'isotope exchange'. 
Experiments conducted with the non-radioactive hydrogen isotope deuterium, have indicated a near-perfect D removal after 24 h of annealing at 520 K under a H$_2$ atmosphere, while a similar removal rate using traditional vacuum annealing would require temperatures of around 700 K. \cite{ahlgren2019hydrogen}

Experimental studies of hydrogen isotope exchange in tungsten have been conducted by e.g. Alimov \textit{et al.} \cite{alimov2011hydrogen}, Roth \textit{et al.}  \cite{roth2013hydrogen}, Barton \textit{et al.} \cite{barton2014deuterium} and Ahlgren \textit{et al.} \cite{ahlgren2019hydrogen}. 
The phenomenon has also been modelled on a macroscopic scale by Hodille \textit{et al.} \cite{hodille2016study} using Macroscopic Rate Equation models, based on experimental results as well as Density Functional Theory (DFT) data on the behaviour of hydrogen near tungsten monovacancies. 
The isotope exchange occurring around a single crystal defect is likely to occur on a timescale of nano- to microseconds, but such scales are far beyond the reach of current DFT methods, while tools such as Rate Equations lack the atomic level insight to the studied mechanism.

In this thesis, we therefore set out to study the isotope exchange mechanism on a microscopic scale, using molecular dynamics to simulate the interactions between hydrogen and various crystallographic defects in tungsten.
We begin with a glance at the background of thermonuclear fusion reactors and the materials used in such devices, before moving on to describe the central computational tools used in this work. 
We continue with an overview of the simulation setups for the various defect types and finally conclude with an analysis of the results.

